%
% ---------------------------------------------------------------
% Copyright (C) 2012-2018 Gang Li
% ---------------------------------------------------------------
%
% This work is the default powerdot-tuliplab style test file and may be
% distributed and/or modified under the conditions of the LaTeX Project Public
% License, either version 1.3 of this license or (at your option) any later
% version. The latest version of this license is in
% http://www.latex-project.org/lppl.txt and version 1.3 or later is part of all
% distributions of LaTeX version 2003/12/01 or later.
%
% This work has the LPPL maintenance status "maintained".
%
% This Current Maintainer of this work is Gang Li.
%
%

\documentclass[
 size=14pt,
 paper=smartboard,  %a4paper, smartboard, screen
 mode=present, 		%present, handout, print
 display=slides, 	% slidesnotes, notes, slides
 style=tuliplab,  	% TULIP Lab style
 pauseslide,
 fleqn,leqno]{powerdot}


\usepackage{cancel}
\usepackage{caption}
\usepackage{stackengine}
\usepackage{smartdiagram}
\usepackage{attrib}
\usepackage{amssymb}
\usepackage{amsmath} 
\usepackage{amsthm} 
\usepackage{mathtools}
\usepackage{rotating}
\usepackage{graphicx}
\usepackage{boxedminipage}
\usepackage{rotate}
\usepackage{calc}
\usepackage[absolute]{textpos}
\usepackage{psfrag,overpic}
\usepackage{fouriernc}
\usepackage{pstricks,pst-3d,pst-grad,pstricks-add,pst-text,pst-node,pst-tree}
\usepackage{moreverb,epsfig,subfigure}
\usepackage{color}
\usepackage{booktabs}
\usepackage{etex}
\usepackage{breqn}
\usepackage{multirow}
\usepackage{natbib}
\usepackage{bibentry}
\usepackage{gitinfo2}
\usepackage{siunitx}
\usepackage{nicefrac}
%\usepackage{geometry}
%\geometry{verbose,letterpaper}
\usepackage{media9}
\usepackage{animate}
%\usepackage{movie15}
\usepackage{auto-pst-pdf}

\usepackage{breakurl}
\usepackage{fontawesome}
\usepackage{xcolor}
\usepackage{multicol}



\usepackage{verbatim}
\usepackage[utf8]{inputenc}
\usepackage{dtk-logos}
\usepackage{tikz}
\usepackage{adigraph}
%\usepackage{tkz-graph}
\usepackage{hyperref}
%\usepackage{ulem}
\usepackage{pgfplots}
\usepackage{verbatim}
\usepackage{fontawesome}


\usepackage{todonotes}
% \usepackage{pst-rel-points}
\usepackage{animate}
\usepackage{fontawesome}

\usepackage{listings}
\lstset{frameround=fttt,
frame=trBL,
stringstyle=\ttfamily,
backgroundcolor=\color{yellow!20},
basicstyle=\footnotesize\ttfamily}
\lstnewenvironment{code}{
\lstset{frame=single,escapeinside=`',
backgroundcolor=\color{yellow!20},
basicstyle=\footnotesize\ttfamily}
}{}


\usepackage{hyperref}
\hypersetup{ % TODO: PDF meta Data
  pdftitle={Predict Future Sales},
  pdfauthor={Lin Peng},
  pdfpagemode={FullScreen},
  pdfborder={0 0 0}
}


% \usepackage{auto-pst-pdf}
% package to show source code

\definecolor{LightGray}{rgb}{0.9,0.9,0.9}
\newlength{\pixel}\setlength\pixel{0.000714285714\slidewidth}
\setlength{\TPHorizModule}{\slidewidth}
\setlength{\TPVertModule}{\slideheight}
\newcommand\highlight[1]{\fbox{#1}}
\newcommand\icite[1]{{\footnotesize [#1]}}

\newcommand\twotonebox[2]{\fcolorbox{pdcolor2}{pdcolor2}
{#1\vphantom{#2}}\fcolorbox{pdcolor2}{white}{#2\vphantom{#1}}}
\newcommand\twotoneboxo[2]{\fcolorbox{pdcolor2}{pdcolor2}
{#1}\fcolorbox{pdcolor2}{white}{#2}}
\newcommand\vpspace[1]{\vphantom{\vspace{#1}}}
\newcommand\hpspace[1]{\hphantom{\hspace{#1}}}
\newcommand\COMMENT[1]{}

\newcommand\placepos[3]{\hbox to\z@{\kern#1
        \raisebox{-#2}[\z@][\z@]{#3}\hss}\ignorespaces}

\renewcommand{\baselinestretch}{1.2}


\newcommand{\draftnote}[3]{
	\todo[author=#2,color=#1!30,size=\footnotesize]{\textsf{#3}}	}
% TODO: add yourself here:
%
\newcommand{\gangli}[1]{\draftnote{blue}{GLi:}{#1}}
\newcommand{\shaoni}[1]{\draftnote{green}{sn:}{#1}}
\newcommand{\gliMarker}
	{\todo[author=GLi,size=\tiny,inline,color=blue!40]
	{Gang Li has worked up to here.}}
\newcommand{\snMarker}
	{\todo[author=Sn,size=\tiny,inline,color=green!40]
	{Shaoni has worked up to here.}}

%%%%%%%%%%%%%%%%%%%%%%%%%%%%%%%%%%%%%%%%%%%%%%%%%%%%%%%%%%%%%%%%%%%%%%%%
% title
% TODO: Customize to your Own Title, Name, Address
%
\title{Predict Future Sales}
\author{
	Lin Peng
	\\
	\\Chongqing University of Posts and Telecomunications
}
\date{\gitCommitterDate}


% Customize the setting of slides
\pdsetup{
% TODO: Customize the left footer, and right footer
rf=\href{http://www.tulip.org.au}{
Last Changed by: \textsc{\gitCommitterName}\ \gitVtagn-\gitAbbrevHash\ (\gitAuthorDate)
},
cf={Group Outlying Aspects Mining},
}


\begin{document}

\maketitle

%\begin{slide}{Overview}
%\tableofcontents[content=sections]
%\end{slide}


%%==========================================================================================
%%
\begin{slide}[toc=,bm=]{Overview}
\tableofcontents[content=currentsection,type=1]
\end{slide}
%%
%%==========================================================================================


\section{Problem Definition}


%%==========================================================================================
%%
\begin{slide}{Description and Evaluation}
	\begin{center}
		\twotonebox{description}{\parbox{.86\textwidth}
			{Using historical sales data, build a forecasting model to predict future sales.
		}}
		\\
		\twotonebox{evaluation}{\parbox{.86\textwidth}
			{Submissions are evaluated by root mean squared error (RMSE). \\True target values are clipped into [0,20] range.
		}}
	\end{center}
	
\end{slide}
%%
%%==========================================================================================

\section{Data Analysis}

%%==========================================================================================
%%
\begin{slide}{Original Data Set}
%Related Work - 导入数据集
\begin{table}
	\small
	\renewcommand\arraystretch{0.7}
	\caption{data set}
	\begin{tabular}{p{5cm} p{8cm} p{5cm}}
		\toprule
		Name & Description & Attribute \\
		\midrule
		sales\_train.csv  & \begin{tabular}[c]{@{}l@{}}
			Training set (historical data from \\ January 2013 to October 2015)
		\end{tabular} & \begin{tabular}[c]{@{}l@{}}
			date\_block\_num,shop\_id,\\item\_id,item\_price,\\data,item\_cnt\_day
		\end{tabular}\\
		\midrule
		test.csv & Test set (Sales forecast for November 2015) & ID,shop\_id,item\_id \\
		\midrule
		sample\_submission.csv & Properly formatted sample submission file & ID,item\_cnt\_month \\
		\midrule
		items.csv & Product's supplemental information & \begin{tabular}[c]{@{}l@{}}
			item\_name,item\_id,\\item\_category\_id 
		\end{tabular}\\
		\midrule
		item\_categories.csv & Project categories' supplemental information & \begin{tabular}[c]{@{}l@{}}
			item\_category\_name,\\item\_category\_id 
		\end{tabular} \\
		\midrule
		shops.csv & store's supplemental information & shop\_name,shop\_id \\
		\bottomrule
	\end{tabular}
\end{table}

\begin{itemize}
	\item
	Training set, with 21807 items, 60 stores.\\
	Test set, 5100 items, 42 stores.
\end{itemize}
\end{slide}
%%
%%==========================================================================================



%%==========================================================================================
%%
\begin{slide}{Missing Values and NaN Values}

\begin{figure}
  \centering
  \selectcolormodel{rgb}
  \includegraphics[width=0.4\textwidth]{figures//trainvalue.eps}
  %\includegraphics[width=0.6\textwidth]{figures//demical.eps}\\
  \caption{Train}\label{fig:demical}
\end{figure}

\begin{figure}
  \centering
  \selectcolormodel{rgb}
  \includegraphics[width=0.4\textwidth]{figures//testvalue.eps}
  %\includegraphics[width=0.6\textwidth]{figures//NBA_team.eps}\\
  \caption{Test}\label{fig:timg}
\end{figure}

\end{slide}
%%
%%==========================================================================================


%%==========================================================================================
%%
\begin{slide}{Outliers and Duplicate Data}

\begin{itemize}
\item
date\_block\_num has 115690 / 3.9\% zeros.
\item
item\_cnt\_day is highly skewed.
\item
Dataset has 6 duplicate rows Warning.
\item
The value of the item\_cnt\_day is less than zero.
\end{itemize}
\begin{figure}
	\centering
	\selectcolormodel{rgb}
	\includegraphics{figures//negative.eps}
	%\includegraphics[width=0.6\textwidth]{figures//NBA_team.eps}\\
	\caption{Train}\label{fig:timg}
\end{figure}

\end{slide}
%%
%%==========================================================================================


%%==========================================================================================
%%
\begin{slide}[toc=,bm=]{Outlier Visualization}
\begin{figure}
	\centering
	\selectcolormodel{rgb}
	\includegraphics[width=0.6\textwidth]{figures//item_cnt_day.eps}
	\includegraphics[width=0.6\textwidth]{figures//item_price.eps}
	%\includegraphics[width=0.6\textwidth]{figures//NBA_team.eps}\\
	\caption{Outliers Data}\label{fig:timg}
\end{figure}
\end{slide}
%%
%%==========================================================================================


%%==========================================================================================
%%
\begin{slide}{Process Training and Test Set Data}
\begin{itemize}
\item
Price less than zero is filled with the mean.
\item
Remove the item\_price over 100000 and the item\_cnt\_day over 1000.
\item
Take advantage of train.drop\_duplicates to remove duplicate columns.
\item
There are several stores that are replicas of each other, changing the training and testing sets to change them to the same store number.
\end{itemize}

\end{slide}
%%
%%==========================================================================================


%%==========================================================================================
%%
\begin{slide}{Process Shops Set}
\begin{itemize}
\item
Shops\_name Decomposition
\begin{itemize}
\item
\smallskip
Break down the city where the store is located.
\item
\smallskip
Break down the store name.
\item
\smallskip
Break down the type of store operation
\end{itemize}
\end{itemize}
\begin{figure}
	\centering
	\selectcolormodel{rgb}
	\includegraphics[width=0.6\textwidth]{figures//process shops.eps}
	%\includegraphics[width=0.6\textwidth]{figures//NBA_team.eps}\\
	\caption{Encode Shops Information}\label{fig:timg}
\end{figure}

\end{slide}
%%
%%==========================================================================================




%%==========================================================================================
%%
\begin{slide}{Process Item\_categories}
%Related Work - Outlying Aspects Mining
\begin{itemize}
	\item
	Item\_category\_name Decomposition
	\begin{itemize}
		\item
		\smallskip
		Break down the title. 
		\item
		\smallskip
		Break down the type of product.
	\end{itemize}
\end{itemize}

\begin{figure}
	\centering
	\selectcolormodel{rgb}
	\includegraphics[width=0.6\textwidth]{figures//process shops.eps}
	%\includegraphics[width=0.6\textwidth]{figures//NBA_team.eps}\\
	\caption{Encode Shops Information}\label{fig:timg}
\end{figure}
\end{slide}
%%
%%==========================================================================================


%%==========================================================================================
%%
\begin{slide}{Process Item}
	
	\begin{itemize}
		\item
		Item\_name Decomposition
	\end{itemize}
	
	\begin{figure}
		\centering
		\selectcolormodel{rgb}
		\includegraphics[width=0.6\textwidth]{figures//processitem.eps}
		\caption{Encode Item Information}\label{fig:timg}
	\end{figure}
	
\end{slide}
%%
%%==========================================================================================

\section{Feature Engineering}

%%==========================================================================================
%%
\begin{slide}{Feature redo - item\_cnt\_month}
	\begin{itemize}
	\item 
	For each month, we create a grid from the unique identifier combination of all stores/products for that month.
	\item
	Monthly sales of this item in this store: item\_cnt\_month.
	\item 
	Using a time trick, append test pairs to the matrix.
	\end{itemize}
	\begin{figure}
		\centering
		\selectcolormodel{rgb}
		\includegraphics[width=0.6\textwidth]{figures//tz.eps}
		\caption{Add item\_cnt\_month}\label{fig:timg}
	\end{figure}

\end{slide}
%%
%%==========================================================================================


%%==========================================================================================
%%
\begin{slide}{Add Variables}
%Challenges (1)
	\begin{figure}
	\centering
	\selectcolormodel{rgb}
	\includegraphics{figures//add varibles.eps}
	\caption{Add Variables}\label{fig:timg}
\end{figure}

\end{slide}
%%
%%==========================================================================================

\section{Modeling and Forecasting}

%%==========================================================================================
%%
\begin{slide}{Challenges (2)}

\begin{itemize}
\item
The XGBRegressor’s fit() method is called to train the model.
\item 
Score: 0.90484
\item 
3283 / 16290
\end{itemize}



\end{slide}
%%
%%==========================================================================================



=================================================================
%


\end{document}

\endinput
